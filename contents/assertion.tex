\begin{savequote}[45mm]
\ascii{Any fool can write code that a computer can understand. Good programmers write code that humans can understand.}
\qauthor{\ascii{- Martin Flower}}
\end{savequote}

\chapter{断言} 
\label{ch:assert-that}

\section{ASSERT\_THAT}

\begin{content}

\ascii{Magellan}只支持一种断言原语:\ascii{ASSERT\_THAT},从而避免用户在\ascii{ASSERT\_EQ/ASSERT\_NE, ASSERT\_TRUE/ASSERT\_FALSE}之间做选择时的困扰,使其断言更加具有统一性。此外,\ascii{ASSERT\_THAT}使得断言更加具有表达力,更加符合语言习惯。

\begin{leftbar}
\begin{c++}[caption={test/hamcrest/CloseToTest.cpp}]
#include <magellan/magellan.hpp>

FIXTURE(CloseToTest)
{
    TEST("double")
    {
        ASSERT_THAT(1.0, close_to(1.0, 0.5));
        ASSERT_THAT(0.5, close_to(1.0, 0.5));
        ASSERT_THAT(1.5, close_to(1.0, 0.5));
    }
};
\end{c++}
\end{leftbar}

\end{content}

\section{EXPECT}

\begin{content}

\begin{table}[!htb]
\resizebox{0.95\textwidth}{!} {
\begin{tabular*}{1.2\textwidth}{@{}ll@{}}
\toprule
xUnit & BDD &  \\
\midrule
\ascii{ASSERT\_THAT}  & \ascii{EXPECT} \\
\bottomrule
\end{tabular*}
}
\caption{Magellan支持的两种断言关键字}
\label{tbl:expect}
\end{table}

\end{content}

\section{Matcher}

\begin{content}

\subsection{Anything}

\begin{table}[!htb]
\resizebox{0.95\textwidth}{!} {
\begin{tabular*}{1.2\textwidth}{@{}ll@{}}
\toprule
匹配器 & 说明 \\
\midrule
\ascii{anything}  & \ascii{总是匹配} \\
\ascii{\_}  & \ascii{anything语法糖} \\
\bottomrule
\end{tabular*}
}
\caption{anything}
\label{tbl:anything-matcher}
\end{table}

\begin{leftbar}
\begin{c++}[caption={test/hamcrest/AnythingTest.cpp}]
#include <magellan/magellan.hpp>

USING_HAMCREST_NS

FIXTURE(AnythingTest)
{
    TEST("should always be matched")
    {
        ASSERT_THAT(1, anything<int>());
        ASSERT_THAT(1u, anything<unsigned int>());
        ASSERT_THAT(1.0, anything<double>());
        ASSERT_THAT(1.0f, anything<float>());
        ASSERT_THAT(false, anything<bool>());
        ASSERT_THAT(true, anything<bool>());
        ASSERT_THAT(nullptr, anything<std::nullptr_t>());
    }

    TEST("should support _ as syntactic sugar")
    {
        ASSERT_THAT(1u, _(int));
        ASSERT_THAT(1.0f, _(float));
        ASSERT_THAT(false, _(int));
        ASSERT_THAT(nullptr, _(std::nullptr_t));
    }
};
\end{c++}
\end{leftbar}

\subsection{比较器}

\begin{table}[!htb]
\resizebox{0.95\textwidth}{!} {
\begin{tabular*}{1.2\textwidth}{@{}ll@{}}
\toprule
匹配器 & 说明 \\
\midrule
\ascii{eq}  & \ascii{相等} \\
\ascii{ne}  & \ascii{不相等} \\
\ascii{lt}  & \ascii{小于} \\
\ascii{gt}  & \ascii{大于} \\
\ascii{le}  & \ascii{小于或等于} \\
\ascii{ge}  & \ascii{大于或等于} \\
\bottomrule
\end{tabular*}
}
\caption{比较的Matcher}
\label{tbl:comparable-matcher}
\end{table}

\begin{leftbar}
\begin{c++}[caption={test/hamcrest/ComparableTest.cpp}]
#include <magellan/magellan.hpp>

USING_HAMCREST_NS

FIXTURE(EqualToTest)
{
    TEST("should allow compare to integer")
    {
        ASSERT_THAT(0xFF, eq(0xFF));
        ASSERT_THAT(0xFF, is(eq(0xFF)));

        ASSERT_THAT(0xFF, is(0xFF));
        ASSERT_THAT(0xFF == 0xFF, is(true));
    }

    TEST("should allow compare to bool")
    {
        ASSERT_THAT(true, eq(true));
        ASSERT_THAT(false, eq(false));
    }

    TEST("should allow compare to string")
    {
        ASSERT_THAT("hello", eq("hello"));
        ASSERT_THAT("hello", eq(std::string("hello")));
        ASSERT_THAT(std::string("hello"), eq(std::string("hello")));
    }
};

FIXTURE(NotEqualToTest)
{
    TEST("should allow compare to integer")
    {
        ASSERT_THAT(0xFF, ne(0xEE));

        ASSERT_THAT(0xFF, is_not(0xEE));
        ASSERT_THAT(0xFF, is_not(eq(0xEE)));
        ASSERT_THAT(0xFF != 0xEE, is(true));
    }

    TEST("should allow compare to boolean")
    {
        ASSERT_THAT(true, ne(false));
        ASSERT_THAT(false, ne(true));
    }

    TEST("should allow compare to string")
    {
        ASSERT_THAT("hello", ne("world"));
        ASSERT_THAT("hello", ne(std::string("world")));
        ASSERT_THAT(std::string("hello"), ne(std::string("world")));
    }
};
\end{c++}
\end{leftbar}

\subsection{修饰器}

\begin{table}[!htb]
\resizebox{0.95\textwidth}{!} {
\begin{tabular*}{1.2\textwidth}{@{}ll@{}}
\toprule
匹配器 & 说明 \\
\midrule
\ascii{is}  & \ascii{可读性装饰器} \\
\ascii{is\_not}  & \ascii{可读性装饰器} \\
\bottomrule
\end{tabular*}
}
\caption{修饰的Matcher}
\label{tbl:is-matcher}
\end{table}

\begin{leftbar}
\begin{c++}[caption={test/hamcrest/IsNotTest.cpp}]
#include <magellan/magellan.hpp>

USING_HAMCREST_NS

FIXTURE(IsNotTest)
{
    TEST("integer")
    {
        ASSERT_THAT(0xFF, is_not(0xEE));
        ASSERT_THAT(0xFF, is_not(eq(0xEE)));
    }

    TEST("string")
    {
        ASSERT_THAT("hello", is_not("world"));
        ASSERT_THAT("hello", is_not(eq("world")));

        ASSERT_THAT("hello", is_not(std::string("world")));
        ASSERT_THAT(std::string("hello"), is_not(std::string("world")));
    }
};
\end{c++}
\end{leftbar}

\subsection{空指针}

\begin{table}[!htb]
\resizebox{0.95\textwidth}{!} {
\begin{tabular*}{1.2\textwidth}{@{}ll@{}}
\toprule
匹配器 & 说明 \\
\midrule
\ascii{nil}  & \ascii{空指针} \\
\bottomrule
\end{tabular*}
}
\caption{空指针}
\label{tbl:nil-matcher}
\end{table}

\begin{leftbar}
\begin{c++}[caption={test/hamcrest/NilTest.cpp}]
#include <magellan/magellan.hpp>

USING_HAMCREST_NS

FIXTURE(NilTest)
{
    TEST("equal_to")
    {
        ASSERT_THAT(nullptr, eq(nullptr));
        ASSERT_THAT(0, eq(NULL));
        ASSERT_THAT(NULL, eq(NULL));
        ASSERT_THAT(NULL, eq(0));
    }

    TEST("is")
    {
        ASSERT_THAT(nullptr, is(nullptr));
        ASSERT_THAT(nullptr, is(eq(nullptr)));

        ASSERT_THAT(0, is(0));
        ASSERT_THAT(NULL, is(NULL));
        ASSERT_THAT(0, is(NULL));
        ASSERT_THAT(NULL, is(0));
    }

    TEST("nil")
    {
        ASSERT_THAT((void*)NULL, nil());
        ASSERT_THAT((void*)0, nil());
        ASSERT_THAT(nullptr, nil());
    }
};
\end{c++}
\end{leftbar}

\subsection{字符串}

\begin{table}[!htb]
\resizebox{0.95\textwidth}{!} {
\begin{tabular*}{1.2\textwidth}{@{}ll@{}}
\toprule
\ascii{匹配器} & \ascii{说明} \\
\midrule
\ascii{contains\_string}  & \ascii{断言是否包含子串} \\
\ascii{contains\_string\_ignoring\_case}  & \ascii{忽略大小写,断言是否包含子串} \\
\ascii{starts\_with}  & \ascii{断言是否以该子串开头} \\
\ascii{starts\_with\_ignoring\_case}  & \ascii{忽略大小写,断言是否以该子串开头} \\
\ascii{ends\_with}  & \ascii{断言是否以该子串结尾} \\
\ascii{ends\_with\_ignoring\_case}  & \ascii{忽略大小写,断言是否以该子串结尾} \\
\bottomrule
\end{tabular*}
}
\caption{字符串的Matcher}
\label{tbl:string-matcher}
\end{table}

\begin{leftbar}
\begin{c++}[caption={test/hamcrest/StartsWithTest.cpp}]
#include <magellan/magellan.hpp>

USING_HAMCREST_NS

FIXTURE(StartsWithTest)
{
    TEST("case sensitive")
    {
        ASSERT_THAT("ruby-cpp", starts_with("ruby"));
        ASSERT_THAT("ruby-cpp", is(starts_with("ruby")));

        ASSERT_THAT(std::string("ruby-cpp"), starts_with("ruby"));
        ASSERT_THAT("ruby-cpp", starts_with(std::string("ruby")));
        ASSERT_THAT(std::string("ruby-cpp"), starts_with(std::string("ruby")));
    }

    TEST("ignoring case")
    {
        ASSERT_THAT("ruby-cpp", starts_with_ignoring_case("Ruby"));
        ASSERT_THAT("ruby-cpp", is(starts_with_ignoring_case("Ruby")));

        ASSERT_THAT(std::string("ruby-cpp"), starts_with_ignoring_case("RUBY"));
        ASSERT_THAT("Ruby-Cpp", starts_with_ignoring_case(std::string("rUBY")));
        ASSERT_THAT(std::string("RUBY-CPP"), starts_with_ignoring_case(std::string("ruby")));
    }
};
\end{c++}
\end{leftbar}

\subsection{浮点数}

\begin{table}[!htb]
\resizebox{0.95\textwidth}{!} {
\begin{tabular*}{1.2\textwidth}{@{}ll@{}}
\toprule
\ascii{匹配器} & \ascii{说明} \\
\midrule
\ascii{close\_to}  & \ascii{断言浮点数近似等于} \\
\ascii{nan}  & \ascii{断言浮点数不是一个数字} \\
\bottomrule
\end{tabular*}
}
\caption{浮点数的Matcher}
\label{tbl:float-matcher}
\end{table}

\begin{leftbar}
\begin{c++}[caption={test/hamcrest/NanTest.cpp}]
#include <magellan/magellan.hpp>
#include <math.h>

USING_HAMCREST_NS

FIXTURE(IsNanTest)
{
    TEST("double")
    {
        ASSERT_THAT(sqrt(-1.0), nan());
        ASSERT_THAT(sqrt(-1.0), is(nan()));

        ASSERT_THAT(1.0/0.0,  is_not(nan()));
        ASSERT_THAT(-1.0/0.0, is_not(nan()));
    }
};
\end{c++}
\end{leftbar}

\end{content}