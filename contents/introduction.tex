\begin{savequote}[45mm]
\ascii{I'm not a great programmer; I'm just a good programmer with great habits.}
\qauthor{\ascii{- Kent Beck}}
\end{savequote}

\chapter{初识Magellan} 
\label{ch:introduction}

\section{简介}

\begin{content}

\ascii{Magellan}是一个简单的、可扩展的、使用\ascii{C++11}实现的\ascii{xUnit}测试框架,\ascii{Magellan}设计灵感来自于\ascii{Java}社区著名的测试框架\ascii{JUnit}。

\end{content}

\section{安装}

\begin{content}

\subsection{环境准备}

\subsubsection{编译环境}

\ascii{Magellan}目前仅在\ascii{Linux, MAC OS X}系统上测试过,仅支持\ascii{GCC,CLANG}两款编译器。

\begin{table}[!htb]
\resizebox{0.95\textwidth}{!} {
\begin{tabular*}{1.2\textwidth}{@{}ll@{}}
\toprule
编译器 & 最低版本 &  \\
\midrule
\ascii{GCC}  & \ascii{4.8} \\
\ascii{CLANG}  & \ascii{3.4} \\
\bottomrule
\end{tabular*}
}
\caption{支持编译器及其版本}
\label{tbl:supported-compiler}
\end{table}

\subsubsection{安装CMake}

CMake的下载地址是:\href{http://www.cmake.org}{http://www.cmake.org}。以\ascii{Ubuntu}为例,使用\ascii{apt-get}安装\ascii{CMake}。

\begin{leftbar}
\begin{ruby}[caption={安装CMAKE}]
$ sudo apt-get install cmake
\end{ruby}
\end{leftbar}

\subsubsection{安装l0-infra}

\ascii{Magellan}依赖于\ascii{l0-infra},所以必须先安装\ascii{l0-infra}

\begin{leftbar}
\begin{ruby}[caption={安装RVM}]
$ git clone https://gitlab.com/horance/l0-infra.git
$ cd l0-infra
$ mkdir build
$ cd build
$ cmake ..
$ make
$ sudo make install
\end{ruby}
\end{leftbar}

\subsection{安装Magellan}

\begin{leftbar}
\begin{ruby}[caption={安装RVM}]
$ git clone https://gitlab.com/horance/magellan.git
$ cd magellan
$ mkdir build
$ cd build
$ cmake ..
$ make
$ sudo make install
\end{ruby}
\end{leftbar}

\end{content}

\section{破冰之旅}

\begin{content}

\subsection{物理目录}

\begin{leftbar}
\begin{c++}[caption={Quantity物理设计}]
 --quantity
  |-- include
  | |-- quantity
  |-- src
  | |-- quantity  
  |-- test
  | |-- quantity  
  | |-- main.cpp
  |-- CmakeList.txt
\end{c++}
\end{leftbar}

\subsection{\ascii{main}函数}

\begin{leftbar}
\begin{c++}[caption={test/main.cpp}]
#include "magellan/magellan.hpp"

int main(int argc, char** argv)
{
    return magellan::run_all_tests(argc, argv);
}
\end{c++}
\end{leftbar}

\subsection{\ascii{CMAKE}构建脚本}

\begin{leftbar}
\begin{ruby}[caption={quantity/CMakeLists.txt}]
project(quantity)

cmake_minimum_required(VERSION 2.8)

set(CMAKE_CXX_FLAGS "${CMAKE_CXX_FLAGS} -std=c++0x")

include_directories(${CMAKE_CURRENT_SOURCE_DIR}/include)

file(GLOB_RECURSE all_files 
src/*.cpp
src/*.cc
src/*.c
test/*.cpp
test/*.cc
test/*.c)

add_executable(quantity-test ${all_files})

target_link_libraries(quantity-test magellan l0-infra)
\end{ruby}
\end{leftbar}

\subsection{构建\ascii{Quantity}}

\begin{leftbar}
\begin{ruby}[caption={构建Quantity}]
$ mkdir build
$ cd build
$ cmake ..
$ make
\end{ruby}
\end{leftbar}

\subsection{执行测试}

\begin{ruby}[caption={执行测试}]
$ ./quantity-test

[==========] Running 0 test cases.
[----------] 0 tests from All Tests
[----------] 0 tests from All Tests

[==========] 0 test cases ran.
[  TOTAL   ] PASS: 0  FAILURE: 0  ERROR: 0  TIME: 0 us

\end{ruby}

\section{体验Magellan}

\subsection{第一个测试用例}

\begin{leftbar}
\begin{c++}[caption={test/quantity/LengthTest.cpp}]
#include <magellan/magellan.hpp>
#include "quantity/Length.h"

USING_HAMCREST_NS

FIXTURE(LengthTest)
{
    TEST("1 FEET should equal to 12 INCH")
    {
        ASSERT_THAT(Length(1, FEET), eq(Length(12, INCH)));
    }
};
\end{c++}
\end{leftbar}

使用\ascii{Magellan},只需要包含\ascii{magellan.hpp}一个头文件即可。\ascii{Magellan}使用\ascii{Hamcrest}的断言机制,使得断言更加统一、自然,且具有良好的扩展性;使用\ascii{USING\_HAMCREST\_NS},从而可以使用\ascii{eq}代替\ascii{hamcrest::eq},简短明确;除非出现名字冲突,否则推荐使用简写的\ascii{Matcher}。

\subsection{Length实现}

\begin{tabular}{@{}p{\dimexpr0.45\linewidth-\tabcolsep} 
                 | p{\dimexpr0.45\linewidth-\tabcolsep}@{}}
\begin{c++}[caption={include/quantity/Length.h}]
#include "quantity/Amount.h"

enum LengthUnit
{
    INCH = 1,
    FEET = 12 * INCH,
};

struct Length
{
    Length(Amount amount, LengthUnit unit);

    bool operator==(const Length&) const;
    bool operator!=(const Length&) const;

private:
    const Amount amountInBaseUnit;
};
\end{c++}
&
\begin{c++}[caption={src/quantity/Length.cpp}]
#include "quantity/Length.h"

Length::Length(Amount amount, LengthUnit unit)
  : amountInBaseUnit(unit * amount)
{
}

bool Length::operator==(const Length& rhs) const
{
    return amountInBaseUnit == rhs.amountInBaseUnit;
}

bool Length::operator!=(const Length& rhs) const
{
    return !(*this == rhs);
}
\end{c++}
\end{tabular}

\subsection{构建\ascii{Quantity}}

\begin{leftbar}
\begin{ruby}[caption={构建Quantity}]
$ cd build
$ cmake ..
$ make
\end{ruby}
\end{leftbar}

\subsection{执行测试}

\begin{ruby}[caption={执行测试}]
$ ./quantity-test

[==========] Running 1 test cases.
[----------] 1 tests from All Tests
[----------] 1 tests from LengthTest
[ RUN      ] LengthTest::1 FEET should equal to 12 INCH
[       OK ] LengthTest::1 FEET should equal to 12 INCH(13 us)
[----------] 1 tests from LengthTest

[----------] 1 tests from All Tests

[==========] 1 test cases ran.
[  TOTAL   ] PASS: 1  FAILURE: 0  ERROR: 0  TIME: 13 us
\end{ruby}

\end{content}
