\begin{savequote}[45mm]
\ascii{Write programs for people first, computers second.}
\qauthor{\ascii{- Steve McConnell}}
\end{savequote}

\chapter{用例设计} 
\label{ch:test-case}

\section{Fixture}

\begin{content}

\subsection{Fixture}

在编写测试时,最费时的部分之一是将整个场景设置成某个已知的状态,并在测试结束后将其复原到初始状态。这个已知的状态称为测试的\ascii{Fixture}。 

\begin{leftbar}
\begin{c++}[caption={test/quantity/LengthTest.cpp}]
#include <magellan/magellan.hpp>

FIXTURE(LengthTest)
{
};
\end{c++}
\end{leftbar}

\ascii{FIXTURE}的参数可以是任意的\ascii{C/C++}标识符。一般而言,将其命名为\ascii{CUT(Class Under Test)}的名字即可。

\subsection{独立的Fixture}

使用\ascii{Setup/Teardown}分别完成场景的设置,并在测试结束后将其复原到初始状态。其中,\ascii{SETUP/TEARDOWN}为可选,如果未配置则默认实现为空。

\begin{leftbar}
\begin{c++}[caption={test/quantity/LengthTest.cpp}]
#include <magellan/magellan.hpp>

FIXTURE(LengthTest)
{
    Length length;

    SETUP()
    {}

    TESTDOWN()
    {}

    TEST("length test1")
    {}

    TEST("length test2")
    {}
};
\end{c++}
\end{leftbar}

执行序列:
\begin{enum}
  \eitem{\ascii{Length}构造函数}
  \eitem{\ascii{SETUP}}
  \eitem{\ascii{length test1}}
  \eitem{\ascii{TESTDOWN}}
  \eitem{\ascii{Length}析构函数}
  \eitem{\ascii{Length}构造函数}
  \eitem{\ascii{SETUP}}
  \eitem{\ascii{length test2}}
  \eitem{\ascii{TESTDOWN}}
  \eitem{\ascii{Length}析构函数}
\end{enum}

\subsection{BDD风格}

\begin{table}[!htb]
\resizebox{0.95\textwidth}{!} {
\begin{tabular*}{1.2\textwidth}{@{}ll@{}}
\toprule
xUnit & BDD &  \\
\midrule
\ascii{FIXTURE}  & \ascii{CONTEXT} \\
\ascii{SETUP}  & \ascii{BEFORE} \\
\ascii{TEARDOWN}  & \ascii{AFTER} \\
\bottomrule
\end{tabular*}
}
\caption{Magellan支持的两种\ascii{Fixture}风格}
\label{tbl:xunit-bdd-style}
\end{table}

\subsection{共享的Fixture}

假设存在如下的场景,每个\ascii{Test}都需要初始化比较耗时的资源,举例如数据库连接、开启浏览器等等。如果一个\ascii{Test}自己完成启动和关闭,测试的时间会拖的很长,给持续集成带来了很大的困难。

为此需要在每组不会冲突的\ascii{Test}中共享一个浏览器窗口、或数据库的链接,为它们设计共享的\ascii{Fixture}。

\begin{leftbar}
\begin{c++}[caption={test/quantity/LengthTest.cpp}]
#include <magellan/magellan.hpp>

FIXTURE(LengthTest)
{
    Length length;

    BEFORE_CLASS()
    {}

    AFTER_CLASS()
    {}

    BEFORE()
    {}

    AFTER()
    {}

    TEST("length test1")
    {}

    TEST("length test2")
    {}
};
\end{c++}
\end{leftbar}

\ascii{test1和test2}的执行序列:
\begin{enum}
  \eitem{\ascii{BEFORE\_CLASS}}
  \eitem{\ascii{Length}构造函数}
  \eitem{\ascii{BEFORE}}
  \eitem{\ascii{length test1}}
  \eitem{\ascii{AFTER}}
  \eitem{\ascii{Length}析构函数}
  \eitem{\ascii{Length}构造函数}
  \eitem{\ascii{BEFORE}}
  \eitem{\ascii{length test2}}
  \eitem{\ascii{AFTER}}
  \eitem{\ascii{Length}析构函数}
  \eitem{\ascii{AFTER\_CLASS}}
\end{enum}

\subsection{全局的Fixture}

有时候需要在所有用例启动之前完成一次性的全局性的配置,在所有用例运行完成之后完成一次性的清理工作。\ascii{Magellan}则使用\ascii{BeforeAll}和\ascii{AfterAll}两个关键字来支持这样的特性的。

\begin{leftbar}
\begin{c++}[caption={test/DatabaseTest.cpp}]
#include <magellan/magellan.hpp>

BEFORE_ALL("before all 1")
{
}

BEFORE_ALL("before all 2")
{
}

AFTER_ALL("after all 1")
{
}

AFTER_ALL("after all 2")
{
}
\end{c++}
\end{leftbar}

\ascii{BeforeAll}和\ascii{AfterAll}向系统注册\ascii{Hook}即可,\ascii{Magellan}便能自动地发现它们,并执行它们。犹如\ascii{C/C++}不能保证各源文件中全局变量初始化的顺序一样,避免在源文件之间的\ascii{BeforeAll}设计不合理的依赖关系。

\begin{leftbar}
\begin{c++}[caption={test/quantity/LengthTest.cpp}]
#include <magellan/magellan.hpp>

FIXTURE(LengthTest)
{
    Length length;

    BEFORE_CLASS()
    {}

    AFTER_CLASS()
    {}

    BEFORE()
    {}

    AFTER()
    {}

    TEST("length test1")
    {}

    TEST("length test2")
    {}
};
\end{c++}
\end{leftbar}

\begin{leftbar}
\begin{c++}[caption={test/quantity/VolumeTest.cpp}]
#include <magellan/magellan.hpp>

FIXTURE(VolumeTest)
{
    Volume volume;

    BEFORE_CLASS()
    {}

    AFTER_CLASS()
    {}

    BEFORE()
    {}

    AFTER()
    {}

    TEST("volume test1")
    {}

    TEST("volume test1")
    {}
};
\end{c++}
\end{leftbar}

\ascii{Magellan可能}的执行序列:
\begin{enum}
  \eitem{\ascii{before all 1}}
  \eitem{\ascii{before all 2}}
  \eitem{\ascii{LengthTest::BEFORE\_CLASS}}
  \eitem{\ascii{Length}构造函数}
  \eitem{\ascii{LengthTest::BEFORE}}
  \eitem{\ascii{length test1}}
  \eitem{\ascii{LengthTest::AFTER}}
  \eitem{\ascii{Length}析构函数}
  \eitem{\ascii{Length}构造函数}
  \eitem{\ascii{LengthTest::BEFORE}}
  \eitem{\ascii{length test2}}
  \eitem{\ascii{LengthTest::AFTER}}
  \eitem{\ascii{Length}析构函数}
  \eitem{\ascii{LengthTest::AFTER\_CLASS}}
  \eitem{\ascii{VolumeTest::BEFORE\_CLASS}}
  \eitem{\ascii{Volume}构造函数}
  \eitem{\ascii{LengthTest::BEFORE}}
  \eitem{\ascii{volume test1}}
  \eitem{\ascii{LengthTest::AFTER}}
  \eitem{\ascii{Volume}析构函数}
  \eitem{\ascii{Volume}构造函数}
  \eitem{\ascii{LengthTest::BEFORE}}
  \eitem{\ascii{volume test2}}
  \eitem{\ascii{LengthTest::AFTER}}
  \eitem{\ascii{Volume}析构函数}
  \eitem{\ascii{VolumeTest::AFTER\_CLASS}}
  \eitem{\ascii{after all 2}}
  \eitem{\ascii{after all 1}}
\end{enum}

\end{content}

\section{Test}

\begin{content}

\subsection{自动标识}

\ascii{Magellan}能够自动地实现测试用例的标识功能,用户可以使用字符串来解释说明测试用例的意图,使得用户在描述用例时更加自然和方便。

\begin{leftbar}
\begin{c++}[caption={test/quantity/LengthTest.cpp}]
#include <magellan/magellan.hpp>
#include "quantity/length/Length.h"

USING_HAMCREST_NS

FIXTURE(LengthTest)
{
    TEST("1 FEET should equal to 12 INCH")
    {
        ASSERT_THAT(Length(1, FEET), eq(Length(12, INCH)));
    }

    TEST("1 YARD should equal to 3 FEET")
    {
        ASSERT_THAT(Length(1, YARD), eq(Length(3, FEET)));
    }

    TEST("1 MILE should equal to 1760 YARD")
    {
        ASSERT_THAT(Length(1, MILE), eq(Length(1760, YARD)));
    }
};
\end{c++}
\end{leftbar}

\subsection{面向对象}

\ascii{Magellan}实现\ascii{xUnit}时非常巧妙,使得用户设计用例时更加面向对象。

\begin{leftbar}
\begin{c++}[caption={test/robot-cleaner/RobotCleanerTest.cpp}]
#include "magellan/magellan.hpp"
#include "robot-cleaner/RobotCleaner.h"
#include "robot-cleaner/Position.h"
#include "robot-cleaner/Instructions.h"

USING_HAMCREST_NS

FIXTURE(RobotCleanerTest)
{
    TEST("at the beginning, the robot should be in at the initial position")
    {
        RobotCleaner robot;

        ASSERT_THAT(robot.getPosition(), is(Position(0, 0, NORTH)));
    }

    TEST("left instruction: 1-times")
    {
        RobotCleaner robot;

        robot.exec(left());
        
        ASSERT_THAT(robot.getPosition(), is(Position(0, 0, WEST)));
    }

    TEST("left instruction: 2-times")
    {
        RobotCleaner robot;
        
        robot.exec(left());
        robot.exec(left());
        
        ASSERT_THAT(robot.getPosition(), is(Position(0, 0, SOUTH)));
    }
};
\end{c++}
\end{leftbar}

\subsubsection{消除重复}

用例之间存在重复代码,为了改善设计,首先将所有用例使用的\ascii{RobotCleaner}对象直接定义在类体里即可。但在运行时,每个用例可得到独立的\ascii{RobotCleaner}实例。

\begin{leftbar}
\begin{c++}[caption={test/robot-cleaner/RobotCleanerTest.cpp}]
#include "magellan/magellan.hpp"
#include "robot-cleaner/RobotCleaner.h"
#include "robot-cleaner/Position.h"
#include "robot-cleaner/Instructions.h"

USING_HAMCREST_NS

FIXTURE(RobotCleanerTest)
{
    RobotCleaner robot;

    TEST("at the beginning, the robot should be in at the initial position")
    {
        ASSERT_THAT(robot.getPosition(), is(Position(0, 0, NORTH)));
    }

    TEST("left instruction: 1-times")
    {
        robot.exec(left());        
        ASSERT_THAT(robot.getPosition(), is(Position(0, 0, WEST)));
    }

    TEST("left instruction: 2-times")
    {
        robot.exec(left());
        robot.exec(left());        
        ASSERT_THAT(robot.getPosition(), is(Position(0, 0, SOUTH)));
    }
};
\end{c++}
\end{leftbar}

\subsubsection{函数提取}

提取的相关子函数,可以直接放在\ascii{Fixture}的内部,使得用例与其的距离最近,更加体现类作用域的概念。

\begin{leftbar}
\begin{c++}[caption={test/robot-cleaner/RobotCleanerTest.cpp}]
#include "magellan/magellan.hpp"
#include "robot-cleaner/RobotCleaner.h"
#include "robot-cleaner/Position.h"
#include "robot-cleaner/Instructions.h"

USING_HAMCREST_NS

FIXTURE(RobotCleanerTest)
{
    RobotCleaner robot;

    void WHEN_I_send_instruction(Instruction* instruction)
    {
        robot.exec(instruction);
    }

    void AND_I_send_instruction(Instruction* instruction)
    {
        WHEN_I_send_instruction(instruction);
    }

    void THEN_the_robot_cleaner_should_be_in(const Position& position)
    {
        ASSERT_THAT(robot.getPosition(), is(position));
    }

    TEST("at the beginning")
    {
        THEN_the_robot_cleaner_should_be_in(Position(0, 0, NORTH));
    }

    TEST("left instruction: 1-times")
    {
        WHEN_I_send_instruction(left());
        THEN_the_robot_cleaner_should_be_in(Position(0, 0, WEST));
    }

    TEST("left instruction: 2-times")
    {
        WHEN_I_send_instruction(repeat(left(), 2));
        THEN_the_robot_cleaner_should_be_in(Position(0, 0, SOUTH));
    }

    TEST("left instruction: 3-times")
    {
        WHEN_I_send_instruction(repeat(left(), 3));
        THEN_the_robot_cleaner_should_be_in(Position(0, 0, EAST));
    }

    TEST("left instruction: 4-times")
    {
        WHEN_I_send_instruction(repeat(left(), 4));
        THEN_the_robot_cleaner_should_be_in(Position(0, 0, NORTH));
    }
};
\end{c++}
\end{leftbar}

\end{content}

% \section{Notation}

% \begin{content}

% \end{content}

